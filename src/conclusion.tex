\section{Алгоритм}

\subsection{iSAX}

\begin{frame}{Индексация — iSAX 2.0}
    \begin{orange-box}{iSAX 2.0:}

    \[
        \begin{array}{rcccl}
            C 
            &~=~& (c_{i})^{8}_{i=1}               
            &~=~& \left\{2, 4, 7, 9, 5, 3, 4, 6\right\}; \\
            \overline{C} 
            &~=~& (\overline{c}_{i})^{4}_{i=1}    
            &~=~& \left\{3, 8, 4, 5\right\}; \\
            D^{4} 
            &~=~& (d^{4}_{i})^{4}_{i=1}           
            &~=~& \left\{
                \mathrm{a}, 
                \mathrm{г}, 
                \mathrm{б}, 
                \mathrm{в}
            \right\}; \\
            \SAX(C, 4, 4) 
            &~=~& (b^{2}_{i})^{4}_{i=1}           
            &~=~
            & \left\{
                {(0,0)}, 
                {(1,1)}, 
                {(0,1)}, 
                {(1,0)} 
            \right\}; \\
            \iSAX(C, 4, 4) 
            &~=~& (s_{i})^{4}_{i=1}       
            &~=~       
            & \left\{
                {(0)}^{2}, 
                {(1,1)}^{4}, 
                {(1)}^{2}, 
                {(1,0)}^{4}
            \right\}; \\
        \end{array}
    \]
        
        
    \end{orange-box}
\end{frame}

\subsection{Поиск и индексация}

\begin{frame}{Итоговый алгоритм}
    \scriptsize
    Для испытуемого потока видео:
    \begin{enumerate}
        \item выделить аномалии видео (границы съёмок) 
            с~помощью сравнения скользящих 
            средних оценок сигнала;
        \item построить дескрипторы съёмок:
        \begin{enumerate}
            \item вычислить относительные длины съёмок,
            \item вычислить характеристики граничных кадров;
        \end{enumerate}
        \item найти дескрипторы съёмок в индексе
              с помощью модифицированного алгоритма $\iSAX{2.0}$:
        \begin{enumerate}
            \item выполнить кусочно-агрегатную аппроксимацию
                  дескрипторов съёмок,
            \item выполнить поиск по индексному дереву;
        \end{enumerate}
        \item если дескрипторы съёмок искомого видео найдены:
        \begin{itemize}
            \item сравнить характеристики 
                  граничных кадров съёмок,
            \item в случае успеха 
                  пометить искомое видео как дубликат;
        \end{itemize}
        \item в противном случае: 
        \begin{enumerate}
            \item добавить испытуемое видео 
                в индекс с помощью модифицированного 
                алгоритма $\iSAX{2.0}$,
            \item перестроить индексное дерево.
        \end{enumerate}
    \end{enumerate}
\end{frame}

\begin{frame}{Свойства алгоритма}
    \begin{orange-box}{Достоинства:}
        \small
        \begin{itemize}
        \item алгоритм работает в режиме реального времени;
        \item алгоритм может быть применен 
        к видео различных типов без полного 
        перестроения индекса;
        \item благодаря тому, что при поиске не учитываются 
        пространственные характеристики видео, 
        алгоритм неприхотлив к вычислительным ресурсам.
        \end{itemize}
    \end{orange-box}
    \vspace{12pt}
    \begin{orange-box}{Недостатки:}
        \small
        \begin{itemize}
            \item   не учитываются 
                    пространственные характеристики видео;
            \item   не учитывается
                    семантическая информация видео;
            \item   требуется настройка размера скользящих 
                    окон при поиске аномалий;
            \item   требуется настройка коэффициента дискретизации 
                    при построении индекса и поиску по индексу.
        \end{itemize}
    \end{orange-box}
\end{frame}

\subsection{Итог}

\begin{frame}{Итог}
   \begin{orange-box}{На защиту выносятся}
        \begin{itemize}
            \item 
                метод сравнения и~поиска нечетких дубликатов видео, 
                который с~помощью разбиения видео на события 
                сводит задачу к сравнению и~поиску временных рядов; 
            \item
                теорема о нечетких дубликатах видео;
            \item
                способ описания процесса обработки видео
                с помощью декларативного языка фильтров, 
                пример приведён ранее~— на рисунке;
            \item 
                грамматика декларативного языка фильтров;
            \item
                алгоритм сегментации видео
                на основе сравнения скользящих 
                средних оценок сигнала.
        \end{itemize}
    \end{orange-box}
\end{frame}
