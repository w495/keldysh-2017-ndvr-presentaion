
\subsection{Виды}

\begin{frame}[allowframebreaks]{Поиск на основе сцен}

    \orangebox{Кадр\ ---~{\it frame}, фотографический кадр}
    {\footnotesize
        \begin{itemize}
            \item[${\color{pacificorange} \Leftarrow}$]
                отдельная статическая картинка;
            \item[${\color{pacificorange} \Leftarrow}$]
                обозначим $\videoframe$.
        \end{itemize}
    }

    \vspace{12pt}
    \zgreenbox{Cъемка\ ---~{\it shot}, кинематографический кадр}
    {\footnotesize
        \begin{itemize}
            \item[${\color{zdarkgreen} \Leftarrow}$]
                множество фотографических кадров,
                единство процесса съемки;
            \item[${\color{zdarkgreen} \Leftarrow}$]
                обозначим $\videoshot$,
                $\videoframe \in \videoshot$;
            \item[${\color{zdarkgreen} \Leftarrow}$]
                часто называют <<сценой>>, 
                далее будем рассматривать, $\videoshot$, назвая сценой;
        \end{itemize}
    }
    \vspace{12pt}
    \zbluebox{Сцена\ ---~{\it scene}, монтажный кадр}
    {\footnotesize
        \begin{itemize}
            \item[${\color{zdarkblue} \Leftarrow}$]
                множество фотографических кадров,
                единство места и времени;
            \item[${\color{zdarkblue} \Leftarrow}$]
                обозначим $\videoscene$,
                $\videoframe \in \videoshot \subset \videoscene$.
        \end{itemize}
    }

    \zgreenbox{Сцена как <<съемка>>, кинематографический кадр}
    {
        ---~совокупность множества фотографических кадров $\videoframe$
        внутри временной области $\videoline$, кадры,
        которой $\videoframe_{\color{red} \videoshot, i}$
        значительно отличается от кадров соседних областей.
        \[
            \videoshot =
                \{
                    \videoframe_{\color{red} \videoshot, i}
                        | \videoframediff(\videoframe_{\color{red} \videoshot, i},
                            \videoframe_{\color{red} \videoshot, j})
                                < {\color{red} \varepsilon},
                            \videoframe_{\color{red} \videoshot, i},
                            \videoframe_{\color{red} \videoshot, j}
                            \in \videoline
                \}
        \]\[
            \videoframediff \text{\footnotesize \ ---~функция разности кадров}.
        \]
    }

    Аналогично можно ввести определение <<звуковой сцены>>,
    предварительно разделив звуковой сигнал на отсчеты.
\end{frame}


\begin{frame}{Аномалии}

    \begin{gray-box}{Аномалия}
        Аномалия — скачкообразное изменение свойств наблюдаемого ряда 
        в заранее неизвестный момент времени. 
        Иногда аномалию называют «разладкой».
    \end{gray-box}
    \vspace{12pt}
    \begin{blue-box}{Моменты событий:}
        моменты времени, в~которых наблюдались аномалии
    \end{blue-box}
    \vspace{12pt}
    \begin{orange-box}{Утверждение о событиях}
        Моменты событий совпадают с моментами смены съёмок. 
    \end{orange-box}

\end{frame}



\begin{frame}{Аномалии}
    \begin{blue-box}{}
        Пусть даны два видео — $V(t)$ и $W(t)$.
        Считаем, что видео является частным случаем временного ряда.
    \end{blue-box}
    \vspace{12pt}
    \begin{blue-box}{Аномалия}
        Из точек временного ряда $V(t)$, в которых наблюдались аномалии, 
        построим временной ряд $(ev_{i})^{k}_{i=1}$.
        \[
            \left\lbrace
                \begin{array}{lcl}
                    (ev_{i})^{k}_{i=1} & = &  ev_1, ev_2, \ldots, ev_k; \\
                    \left\lbrace ev_1, ev_2, \ldots, ev_k  \right\rbrace & \subset & V 
                \end{array}
            \right.
        \]
        Для $W(t)$ аналогичным образом построим  $(ew_{i})^{n}_{i=1}$.
    \end{blue-box}
\end{frame}



\begin{frame}{Нечёткие дубликаты}
    \begin{orange-box}{Одинаковые явления}
    $V$ и $W$ выражают одинаковую
    последовательность явлений, если существует $\tser{x}{m}$, 
    такой что $ \tset{x}{m} \subset \tset{ev}{k} $ 
    и~$ \tset{x}{m} \subset \tset{ew}{n}$.
    \[
        V \esim  W \Leftrightarrow  
        \exists \tser{x}{m}; \quad \left\lbrace
            \begin{array}{lclll}
                \tset{x}{m} & \subset 
                    & \tset{ev}{k} 
                    & \subset & V; \\
                \tset{x}{m} & \subset 
                    & \tset{ew}{n} 
                    & \subset & W 
            \end{array}
        \right.
    \]
    \end{orange-box}
        \vspace{12pt}
    \begin{gray-box}{Нечёткие дубликаты}
        $V$ и $W$ — нечёткие дубликаты друг друга,
        если существует $\tser{x}{m}$, 
        такой что $ \tset{x}{m} \subset \tset{v}{k} $ 
        и~$ \tset{x}{m} \subset \tset{w}{n}$.
        \[
            V \ndsim  W \Leftrightarrow 
            \exists \tser{x}{m}; \quad \left\lbrace
                \begin{array}{lclll}
                    \tset{x}{m} & \subset 
                        & \tset{v}{k} 
                        & \subset & V; \\
                    \tset{x}{m} & \subset 
                        & \tset{w}{n} 
                        & \subset & W 
                \end{array}
            \right.
        \]
    \end{gray-box}
\end{frame}


\begin{frame}{Теорема о нечётких дубликатах}

   \begin{orange-box}{Одинаковые явления}
        Нечеткие дубликаты видео выражают 
        одинаковую последовательность явлений.
        \[
             V \esim  W \Rightarrow  V \ndsim  W
        \]
    \end{orange-box}

\end{frame}


% \begin{frame}{Теорема о дубликатах}

%     \[
%         \left\lbrace
%             \begin{array}{lcl}
%                 (ev_{i})^{k}_{i=1} & = &  ev_1, ev_2, \ldots, ev_k; \\
%                 \left\lbrace ev_1, ev_2, \ldots, ev_k  \right\rbrace & \subset & V 
%             \end{array}
%         \right.
%     \]

%         \begin{gray-box}{Видео — последовательность фактов или событий}
%         \begin{itemize}
%             \item События развиваются во времени.
%             \item Свойства событий
%             — {\it пространственная} характеристика видео,
%             \item продолжительность и порядок фактов — {\it временная}.
%         \end{itemize}%
%     \end{gray-box}
% \end{frame}



% \begin{frame}{Приложение: алгоритм поиска нечетких дубликатов}
%     \begin{itemize} \footnotesize
%         \item $\nv$\ ---~новое видео;
%         \item $\setsv = \{\sv[1], \sv[2], \dots, \sv[n]\}$\ ---~исходные видео:
%             \begin{itemize}
%                 \item[${\color{zdarkblue}\leftarrow}$]
%                     {\scriptsize  $\setsv$ может быть пустым};
%                 \item[${\color{zdarkblue}\leftarrow}$]
%                     {\scriptsize для непустого $\setsv$
%                         вычислены дескрипторы сцен элементов.}
%             \end{itemize}
%         \item[1.] Сравниваем дескриптор каждой
%             сцены $\videoshot_{\color{red}\nv,i}$ из $\nv$\
%             с дескриптором каждой сцены
%             $\videoshot_{\color{red}\sv[k],j}$ из $\sv[k]$ в $L_2$.
%         \item[2.] Если дескрипторы совпали $\nv$ c дескрипторами $\sv[k]$.
%             на~некотором временном промежутке ,
%             то считаем эту часть $\nv$\ ---~дубликатом $\sv[k]$,
%             \begin{itemize}
%                 \item[] {\scriptsize несовпавшие
%                     части $\nv$\ помещаем в $\setsv$}.
%             \end{itemize}
%         \item[3.] Если дескрипторы не совпали, то считаем $\nv$\ уникальным и
%             добавляем в $\setsv$.
%     \end{itemize}
% \end{frame}


% \begin{frame}{Результаты и перспективы}

%     \orangebox{Предложен подход}{
%         \vspace{0.5em}
%         \begin{itemize}
%             \item[$\rhd$] относительные длины,
%             \item[$\rhd$] выравнивания,
%             \item[] \large $\color{zdarkgreen}\Rightarrow$ дескриптор \sout{сцен} {\it фактов в видео};
%         \end{itemize}
%         \vspace{0.5em}
%     }
%     \vspace{1.5em}
%     \graybox{Проведены эксперименты (17 тыс. фильмов)}{
%         \vspace{0.5em}
%         \begin{itemize}
%             \item точность $ = 0.8$;
%             \item полнота $ = 0.7$;
%             \item ложноотрицательные оценки;
%             \item[$\color{red}\blacklozenge$] требуется более
%                 детальное практическое сравнение с~существующими методами.
%         \end{itemize}
%         \vspace{0.5em}
%     }

% \end{frame}

