% !TeX encoding = UTF-8


\begin{frame}{Улучшения статического порога}
    
    \begin{orange-box}{Проблема:}
        \begin{itemize}
        \item большой порог для плавного видео:
            \begin{itemize}
                \item события найдены не будут;
            \end{itemize}
        \item маленький порог для динамичного видео:
            \begin{itemize}
                \item много ложных срабатываний;
            \end{itemize}
        \item если видео имеет плавные и динамичные участки:
            \begin{itemize}
                \item порог подобрать невозможно.
            \end{itemize}
        \end{itemize}
    \end{orange-box}
    \vspace{1.5em}
    \begin{grass-green-box}{Решение:}
        \begin{itemize}
            \item Учитывать «динамичность» видео:
            \begin{enumerate}
                \item Масштабировать разницы кадров:
                \begin{itemize}
                    \item нормировать по окрестности.
                \end{itemize}
                \item Учитывать среднюю величину и дисперсию.
            \end{enumerate}
        \end{itemize}
    \end{grass-green-box}
    
\end{frame}

\subsubsection*{Нормировка по размаху}

\begin{frame}{Нормировка по размаху}

    \begin{gray-box}{Нормировка разницы кадров}
        \begin{enumerate}
            \item Выберем размер скользящего окна (вектора задержек) — $s$.
            \item Для каждого кадра из последовательности:
            \begin{enumerate}
                \item Вычислим максимальное и минимальное значение разницы 
                кадров на этом окне.
                \item Вычислим размах разниц для данного скользящего окна.
                \item От текущего значения разницы соседних кадров
                отнимем минимальное значение разницы на скользящем окне,
                и поделим на размах
            \end{enumerate}
            \item Таким образом мы получим нормированное 
            по размаху значение разницы кадров для данного скользящего окна.
            \[
                D_{s,t} = \dfrac
                    {D_{t} - \min \left( D_{t-s} \dots  D_{t} \right)}
                    {\max \left( D_{t-s} \dots  D_{t} \right)
                        - \min \left( D_{t-s} \dots  D_{t} \right)}
            \]  
        \end{enumerate}
    \end{gray-box}
\end{frame}

\begin{frame}[fragile]{Нормировка по размаху: описание на мета-языке}
\begin{lstlisting}[language=FilterPython]
    delay = DelayFilter()      # %{\cmnt Фильтр линейной задержки.}%
    orig  = delay(0)           # %{\cmnt Входящий сигнал без изменения.}%
    shift = ShiftSWFilter()    # %{\cmnt Сдвиг сигнала на один кадр.}%
    diff  = orig - shift       # %{\cmnt Разница соседних кадров.}%
    norm  = NormFilter()       # %{\cmnt Норма сигнала.}%
    
    # %{\cmnt Возвращает скользящее окно размером 400 для каждого кадра}%
    sw = BaseSWFilter(s=400, min_size=2)
    
    sw_max = sw | max  # %{\cmnt Максимум по скользящему окну.}%
    sw_min = sw | min  # %{\cmnt Минимум по скользящему окну.}%
    
    sw_norm = (orig - sw_min) / (sw_max - sw_min) # %{\cmnt Масштабирование.}%
    
    # %{\cmnt Масштабированная разница по скользящему окну}%
    d_sw_filter = diff | abs | norm(l=1) | sw_norm
    
    # %{\cmnt Результирующий фильтр: точки разладок.}%
    result_filter = d_sw_filter | threshold
\end{lstlisting}
\end{frame}


\begin{image-frame}
    [Съёмка с БПЛА над побережьем Тулума]
    {нормировка по размаху (s = 400)}
    \includegraphics[height=7.5cm]%
    {img/video/example/threshold/static/sad-swnorm-400-tulum.pdf}
\end{image-frame}

\begin{image-frame}
    [Съёмка с БПЛА над побережьем Тулума]
    {Нормировка по размаху (s = 200)}
    \includegraphics[height=7.5cm]%
    {img/video/example/threshold/static/sad-swnorm-200-tulum.pdf}
\end{image-frame}

\begin{image-frame}
    [Съёмка с БПЛА над побережьем Тулума]
    {Нормировка по размаху (s = 100)}
    \includegraphics[height=7.5cm]%
    {img/video/example/threshold/static/sad-swnorm-100-tulum.pdf}
\end{image-frame}


\begin{image-frame}{Нормировка по размаху (s = 100), <<Дядя Стёпа~—~милиционер>>}
    \includegraphics[height=7.5cm]%
    {img/video/example/threshold/static/sad-swnorm-100-stepa.pdf}
\end{image-frame}


\begin{image-frame}{Нормировка по размаху (s = 400), <<Дядя Стёпа~—~милиционер>>}
    \includegraphics[height=7.5cm]%
    {img/video/example/threshold/static/sad-swnorm-400-stepa.pdf}
\end{image-frame}


\begin{frame}{Свойства нормировки по размаху}
    \begin{orange-box}{Достоинства:}
        \begin{itemize}
            \item относительная величина порога:
            \begin{itemize}
                \item не нужно подбирать для каждого случая;
                \item можно использовать везде одну и ту же величину.
            \end{itemize}
        \end{itemize}
    \end{orange-box}
    \vspace{1.5em}
    \begin{blue-box}{Недостатки:}
        \begin{itemize}
            \item нужно подбирать размер окна:
            \begin{itemize}
                \item при малом размере окна — 
                нормируем для паразитных шумов
                \item при большом размере 
                — пропускаем разладки.
            \end{itemize}
        \end{itemize}
    \end{blue-box}
\end{frame}

\subsubsection*{Голосование по размаху}

\begin{frame}{Метод голосований}
    \begin{orange-box}{Достоинства:}
        \begin{itemize}
            \item относительная величина порога:
            \begin{itemize}
                \item не нужно подбирать для каждого случая;
                \item можно использовать везде одну и ту же величину.
            \end{itemize}
        \end{itemize}
    \end{orange-box}
    \vspace{1.5em}
    \begin{blue-box}{Недостатки:}
        \begin{itemize}
            \item нужно подбирать размер окна:
            \begin{itemize}
                \item при малом размере окна — 
                нормируем для паразитных шумов
                \item при большом размере 
                — пропускаем разладки.
            \end{itemize}
        \end{itemize}
    \end{blue-box}
\end{frame}



