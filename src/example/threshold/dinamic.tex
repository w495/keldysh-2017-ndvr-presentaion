
\begin{frame}{Пороговые методы адаптивным порогом}
    \graybox{Как устроены}{
        \begin{itemize}
            \item разница соседних величин (по некоторой норме);
            \item превышения порога — <<аномалия>>.
            \item порог вычисляется динамически (критерий Смирного-Граббса).
        \end{itemize}%
    }
    \vspace{0.5em}
    \orangebox{Плюсы}{
        \begin{itemize}
            \item не требовательны к ресурсам.
        \end{itemize}%
    }
    \vspace{0.5em}
    \bluebox{Минусы}{
        \begin{itemize}
            \item требуется заранее подобрать размер скользящего окна;
            \item не применимо для разных типов видео;
            \item чувствительны к случайным всплескам;
            \item ловят только краткосрочные события;
        \end{itemize}%
    }
\end{frame}


\subsubsection*{Динамичное видео}

\begin{image-frame}{
        <<Дядя Стёпа~—~милиционер>>, \\
        адаптивный порог 
        $D_t > \hat{\mu}_{k} + A \cdot \hat{\sigma}_{k}$
    }
    \includegraphics[width=11cm]
    {img/video/example/threshold/sigma/tsevetkov-stepa-from-00.pdf}
\end{image-frame}

\begin{image-frame}{
        <<Дядя Стёпа~—~милиционер>>,
        адаптивный порог
        $D_t > \hat{\mu}_{k} + A \cdot \hat{\sigma}_{k}$ \\
        (не нашли плавный переход)
    }
    \includegraphics[width=11cm]
    {img/video/example/threshold/sigma/tsevetkov-stepa-from-90.pdf}
\end{image-frame}


\subsubsection*{Спокойное видео}

\begin{image-frame}{
        Съёмка с БПЛА над побережьем Тулума, \\
        адаптивный порог 
        $D_t > \hat{\mu}_{k} + A \cdot \hat{\sigma}_{k}$
    }
    \includegraphics[width=11cm]
    {img/video/example/threshold/sigma/tsevetkov-tulum.pdf}
\end{image-frame}

