% !TeX encoding = UTF-8


\begin{frame}{Пороговые методы cо статическим порогом}
    \begin{gray-box}{Как устроены}
        \begin{itemize}
            \item разница соседних величин (по некоторой норме);
            \item заранее задается некоторый порог;
            \item превышения порога — <<аномалия>>.
        \end{itemize}
    \end{gray-box}
    \vspace{0.5em}
    \begin{orange-box}{Плюсы}
        \begin{itemize}
            \item просты в реализации, втч аппаратной;
            \item не требовательны к ресурсам.
        \end{itemize}%
    \end{orange-box}
    \vspace{0.5em}
    \begin{blue-box}{Минусы}
        \begin{itemize}
            \item требуется заранее знать порог;
            \item не применимо для разных типов видео;
            \item чувствительны к случайным всплескам;
            \item ловят только краткосрочные события.
        \end{itemize}%
    \end{blue-box}
\end{frame}

\subsubsection*{Разница по модулю}

\begin{frame}{Наивный пороговый метод}
    \begin{gray-box}{Разница по модулю}
        \begin{itemize}
            \item нормированная разница яркостей кадров;
            \item векторная норма $L_1$.
        \end{itemize}
    \end{gray-box}
    \vspace{0.5em}
    \begin{gray-box}{}
        \[
        D_t            
        = \left\| F_t - F_{t-1} \right\|_{L_1} 
        = \dfrac{1}{n \cdot m} 
        \sum\limits_{i=0}^{n}
        \sum\limits_{j=0}^{m}
        \left| x_{t,i,j} - x_{t-1,i,j} \right| 
        \]
        \begin{itemize}
            \item $x_{t,i,j}$ 
                — яркость пикселя кадра $F_t$;
            \item $x_{t-1,i,j}$ 
                — яркость пикселя кадра $F_{t-1}$;
            \item $\left| x_{t,i,j} - x_{t-1,i,j} \right|$
                — разница яркостей двух пикселей.
        \end{itemize}
    \end{gray-box}
\end{frame}

\begin{note-frame}
    Это самый простой и очевидный из вех возможных методов.
    
    В качестве исследуемой величины тут используется 
    значение нормированной попиксельной 
    абсолютной разницы яркостей двух соседних кадров.
    В качестве нормы используется векторная норма $L_1$.
    \begin{itemize}
        \item $x_{t,i,j}$ — пиксель кадра $F_t$;
        \item $x_{t-1,i,j}$ — пиксель кадра $F_{t-1}$;
    \end{itemize}
    
    Физический смысл величины 
    $\left| x_{t,i,j} - x_{t-1,i,j} \right|$
    заключается в том, насколько один пиксель отличается от другого.
    
    Если средняя разница пикселей превысит заранее заданное число — порог,
    то мы считаем, что мы нашли <<разладку>>.
    
\end{note-frame}

\begin{image-frame}[<<Дядя Стёпа~—~милиционер>>]{Наивный пороговый метод}
    \includegraphics[height=8.2cm]
    {img/video/example/threshold/static/sad-stepa.pdf}
\end{image-frame}


\subsubsection*{Мета-язык потоковой обработки}

\begin{frame}[fragile]{Мета-язык потоковой обработки}
    \begin{lstlisting}[language=FilterPython]
    delay = DelayFilter()      # %{\cmnt Фильтр линейной задержки.}%
    orig  = delay(0)           # %{\cmnt Входящий сигнал без изменения.}%
    shift = ShiftSWFilter()    # %{\cmnt Сдвиг сигнала на один кадр.}%
    diff = orig - shift        # %{\cmnt Разница соседних кадров.}%
    norm = NormFilter()        # %{\cmnt Норма сигнала.}%
    T_CONST = 0.08             # %{\cmnt Значение порога. Константа.}%
    threshold = orig > T_CONST # %{\cmnt Пороговый фильтр.}%
    
    # %{\cmnt Фильтр нормированной попиксельной абсолютной разницы.}%
    d_filter = diff | abs | norm(l=1)
    
    # %{\cmnt Результирующий фильтр: точки разладок.}%
    result_filter = d_filter | threshold
    \end{lstlisting}
    \begin{itemize}
        \item Оператор <<|>> означает <<конвейер>>.
        \item Фильтры собирается отложено 
                до непосредственного применения.
    \end{itemize}
\end{frame}


\begin{frame}[fragile]{Грамматика языка}
    \begin{lstlisting}[language=FilterPython]
FILTER ::= FILTER_NAME '(' [arglist] ')' | FILTER '|' FILTER | FILTER '|' expr | FILTER op expr

FILTER_NAME ::= POINT_FILTER_NAME | WINDOW_FILTER_NAME

POINT_FILTER_NAME ::= 'Filter' | 'NormFilter' | 'DelayFilter' | 'SignAngleDiff1DFilter' | 'SignAngleDiff2DFilter' | 'SignChangeFilter' | 'ColourFilter'

WINDOW_FILTER_NAME ::= 'BaseSWFilter' | ‘ShiftSWFilter’

op ::= '<' | '>' | '==' | '>=' | '<=' | '<>' | '!=' | '<<' | '>>' | '^' | '+' | '-' | '*' | '**' | '/' | 'in' | 'not' 'in'| 'is' |'is' 'not'

expr ::= 'python expression'
    \end{lstlisting}
\end{frame}


\begin{note-frame}
    Для решения задачи мы разработали мета-язык 
    на базе языка Python.\\
    Свойства мета-языка:
    \begin{enumerate}
        \item Основной сущностью является фильтр:
        \begin{itemize}
            \item по-факту это просто абстрактный тип 
            на языке Python с перегруженными операторами,
            \item фильтры описывают набор действий 
            над последовательностью кадров видео;
        \end{itemize}
        \item Любые прочие объекты приводятся 
        к фильтру и действия 
        с ними осуществляются как с фильтром;
    \end{enumerate}
\end{note-frame}

\begin{note-frame}
    Жизненный цикл фильтра состоит из двух этапов:
    \begin{itemize}
        \item объявление фильтра 
        — при создании фильтра никаких вычислений 
        над последовательностью кадров не происходит,
        \item выполнении фильтра 
        — (вызов метода класса $filter\_objects$) 
        внутри фильтра происходит обработка 
        последовательности объектов: 
        кадров или результатов работы других фильтров;
    \end{itemize}
\end{note-frame}

\begin{note-frame}
    Операции над фильтрами:
    \begin{enumerate}
        \item Любая операция над фильтрами кроме выполнения,
        приводит к созданию нового фильтра:
        \begin{itemize}
            \item на этапе выполнения фильтра будут вычислены 
            аргументы этой операции и выполнена сама операция;
            \item порядок вычисления аргументов операций не определен, и на 
            многоядерных архитектурах аргументы операций могут быть вычислены 
            параллельно;
        \end{itemize}
        \item Основная операция над фильтрами — последовательное их применение — конвейер. Она тоже приводит к созданию нового фильтра.
        \item Прочие операции над фильтрами приводят 
        к изменению последовательности объектов или их свойств
        в процессе выполнения итогового фильтра.
    \end{enumerate}
\end{note-frame}

\subsubsection*{Порог из FFmpeg}

\begin{image-frame}[<<Дядя Стёпа~—~милиционер>>]{Определение <<сцен>> как в FFmpeg}
    \includegraphics[height=8.2cm]
    {img/video/example/threshold/static/ffmpeg-stepa.pdf}
\end{image-frame}


\begin{note-frame}
    FFmpeg — набор свободных библиотек, которые позволяют записывать, обрабатывать и передавать цифровое видео в различных форматах. 
    В том числе, FFmpeg содержит в себе возможность 
    выделения «сцен».
    
    В качестве исследуемой величины тут используется 
    значение наименьшей величины из двух:
    \begin{itemize}
        \item нормированной разницы яркостей двух соседних кадров;
        \item и абсолютной разницы двух последовательных разниц.
    \end{itemize}
    Такой минимум используется для того, чтобы избежать всплесков
    на случай если график сигнала монотонно убывает или возрастает.
    Разница двух последовательных разниц имеет смысл скорости 
    изменения яркости. Малая скорость изменения 
    при значительном изменении говорит о том, 
    что яркость равномерно меняется 
    в некоторой окрестности данного кадра.
    При быстром, но плавном изменении яркостей кадров 
    с точки зрения авторов порога из FFmpeg — разладки нет.
\end{note-frame}


\begin{frame}[fragile]{Описание FFmpeg-порога на мета-языке}
    \begin{lstlisting}[language=FilterPython]
    delay = DelayFilter()      # %{\cmnt Фильтр линейной задержки.}%
    orig  = delay(0)           # %{\cmnt Входящий сигнал без изменения.}%
    shift = ShiftSWFilter()    # %{\cmnt Сдвиг сигнала на один кадр.}%
    diff = orig - shift        # %{\cmnt Разница соседних кадров.}%
    norm = NormFilter()        # %{\cmnt Норма сигнала.}%
    T_CONST = 0.08             # %{\cmnt Значение порога. Константа.}%
    threshold = orig > T_CONST # %{\cmnt Пороговый фильтр.}%
    
    # %{\cmnt Фильтр нормированной попиксельной абсолютной разницы.}%
    d_filter = diff | abs | norm(l=1)
    # %{\cmnt Абсолютная разница двух последовательных разниц.}%
    d_diff_filter = d_filter | diff | abs
    
    # %{\cmnt Минимум между разницей и разницей разниц.}%
    ffmpeg_like = Filter.union(d_filter, d_diff_filter) | min
    
    # %{\cmnt Результирующий фильтр: точки разладок.}%
    result_filter = ffmpeg_like | threshold
    \end{lstlisting}
\end{frame}

\begin{note-frame}
    В приведенном листинге применяется функция $Filter.union$.
    В качестве аргументов передаются фильтры, 
    и выходом функции тоже будет фильтр. 
    $Filter.union$ объединяет выходные последовательности (потоки)
    своих аргументов, так, что единица результирующей  последовательности функции $Filter.union$ содержит 
    в себе соответствующие единицы последовательности аргументов.
    Единицей последовательности в данном случае выступает 
    значение вычисляемой функции для текущего кадра.
    Последовательности аргументов должны быть 
    синхронизированы по времени.
    
    К результату функции $Filter.union$ применяется оператор $\min$.
    В контексте работы с фильтрами это означает, что оператор $\min$
    применяется к каждому элементу выходной последовательности.
    А это в свою очередь означает что будет 
    вычислен минимум двух величин.
    \[
        D_{t}^{ffmpeg} = \min(D_t, \left|D_t - D_{t-1} \right|)
    \]
\end{note-frame}


\begin{frame}[fragile]{FFmpeg: Примечание (1)}
    
    \begin{gray-box}{ffmpeg\_like не совсем ffmpeg:}
        формула и результат совпадают с результатом FFmpeg, 
        с~точностью до~коэффициента $FFMPEG\_CORRECTION$.
    \end{gray-box}

    \begin{lstlisting}[language=FilterPython]
    # %{\cmnt Минимум между разницей и разницей разниц.}%
    ffmpeg_like = Filter.union(d_filter, d_diff_filter) | min
    
    # %{\cmnt Коррекция возникает из-за того что в ffmpeg}% 
    # %{\cmnt цвета представляют целыми числами, без нормировки. }%
    COLOUR_CORRECTION = 3 * 256.0
    
    # %{\cmnt Нормировка взята из исходных кодов ffmpeg. }% 
    FFMPEG_NORM = 100.0
    
    FFMPEG_CORRECTION = COLOUR_CORRECTION / FFMPEG_NORM
    
    # %{\cmnt Как в настоящем FFmpeg.}%
    true_ffmpeg = ffmpeg_like | orig * FFMPEG_CORRECTION
    \end{lstlisting}
\end{frame}

\begin{frame}[fragile]{FFmpeg: Примечание (2)}
Точки линейных склеек «по FFmpeg» можно получить 
с помощью консольной команды:
\begin{lstlisting}[
    language=FFmpegBash,numbers=none
]
ffmpeg -i 'file.mp4' -filter:v "yadif=1:-1:0,select='gt(scene,0.4)',showinfo" -f 'null' -y 'qq' 2> >( grep 'pts_time' | sed -uE 's/.*n:\s*?([0-9]+).*pts:\s*?([0-9]+).*pts_time:\s*?([0-9\.]+).*pos:\s*?([0-9]+).*.*type:\s*?([IPB?]+).*.*mean:\s*?\[(.+)\].*.*stdev:\s*?\[(.+)\].*/n:\1\tpts_time:\3\t\tframe_type:\5\tmean:\6 std:\7/gi' | tee ./out/file-null.log 1>&2);
\end{lstlisting}
Вывод команды:
\begin{lstlisting}[basicstyle=\scriptsize\ttfamily,numbers=none]
    n: 0  pts_time: 8.279  frame_type:P  mean: 20  std:47.0 
    n: 1  pts_time:13.999  frame_type:B  mean: 40  std:71.3 
    n: 2  pts_time:14.159  frame_type:P  mean:180  std:75.7 
    n: 3  pts_time:14.719  frame_type:B  mean:183  std:69.7 
    n: 4  pts_time:14.759  frame_type:B  mean:179  std:70.6 
    n: 5  pts_time:18.319  frame_type:P  mean:187  std:73.2 
    n: 6  pts_time:28.999  frame_type:P  mean: 38  std:53.6 
    n: 7  pts_time:49.119  frame_type:P  mean: 27  std:46.0 
    n: 8  pts_time:61.399  frame_type:P  mean: 57  std:71.3 
    n: 9  pts_time:64.599  frame_type:P  mean: 38  std:58.4 
\end{lstlisting}
    
\end{frame}


\subsubsection*{Сравнение порогов}

\begin{image-frame}{Сравнение порогов}{<<Дядя Стёпа~—~милиционер>>}
    \includegraphics[height=8.2cm]
        {img/video/example/threshold/static/both-stepa.pdf}
\end{image-frame}

\begin{note-frame}
    Сравнивать изложенные методы удобно,
    если их вывести на одном графике. 
    В большинстве случаев для данного временного ряда 
    наивный пороговый метод и порог по FFmpeg совпадают.
    
    Исключение составляет участок с двадцатой по тридцатую секунды.
    В этом месте, наивный метод выявил значимую разницу.
    Порог по FFmpeg эту разницу не обнаружил:
    \begin{itemize}
        \item в первом случае, скорость изменения разницы 
        оказалась недостаточной: при приближении на графике 
        отчетливо видно пологий спуск;
        \item во втором — не хватило величины порога.
    \end{itemize}
    С формальной точки зрения, алгоритм предложенный в FFmpeg,
    лучше наивного метода. Условие минимума разницы 
    и скорости изменения разница отсекает ложные срабатывания.
    Но в данном случае, много зависит от того что считать «разладкой».
\end{note-frame}

\begin{image-frame}
    {Сравнение порогов, секунды с 20 по 40}
    {<<Дядя Стёпа~—~милиционер>>}
    \includegraphics[height=8.2cm]
    {img/video/example/threshold/static/both-stepa-20-40.pdf}
\end{image-frame}


\subsubsection*{Чем плох статический порог}

\begin{frame}{Чем плох статический порог}
    
    \begin{blue-box}{Пример}
        \begin{enumerate}
            \item Видео с плавными переходами:
            \begin{itemize}
                \item например, съёмка с бортовой камеры БПЛА. 
            \end{itemize}
            \item Величину порога как и раньше — $0.08$.
        \end{enumerate}
        \begin{itemize}
            \item В итоге:
            \begin{itemize}
                \item разница кадров всегда меньше порога;
                \item событий «нет».
            \end{itemize}
        \end{itemize}
    \end{blue-box}
    \vspace{1em}
    \begin{gray-box}{В чём проблема:}
        \begin{itemize}
            \item величину порога надо знать заранее;
            \item нельзя использовать везде одну и ту же величину:
            \begin{itemize}
                \item в плавном видео — малые разницы кадров;
                \item в динамичном видео — большие разницы кадров.
            \end{itemize}
        \end{itemize}%
    \end{gray-box}
\end{frame}


\begin{note-frame}
    В качестве плавного видео рассмотрим съёмку с~БПЛА 
    над~побережьем Тулума (Мексика).
    
    Видео снимается с камеры, которая установлена 
    на радиоуправляемый вертолет (квадракоптер).
    БПЛА взлетает с лодки, и медленно поднимается над морем.
    На видео видны сначала море и пустыня с пирамидами майя, 
    потом только пустыня, пирамиды и край неба.
    В процессе подъема камера несколько раз поворачивается 
    из стороны в сторону.
    
    Эти самые повороты камеры, а так тот факт, что в какой-то момент 
    с кадра пропало море, могут быть классифицированы как разладка.
    При пороге $0.08$ ни наивный пороговый метод, 
    ни метод из FFmpeg не зафиксировали никаких событий в этом видео.
\end{note-frame}


\begin{image-frame}{Съёмка с БПЛА над побережьем Тулума ($T_{const} = 0.08$)}
    \includegraphics[height=8.2cm]%
    {img/video/example/threshold/static/sad-ffmpeg-tulum.pdf}
\end{image-frame}

\begin{note-frame}
    На графике средней яркости кадра, и на графиках разностей
    отчетливо видны зазубрины. Это автоматическая калибровка 
    баланса белого в камере. Калибровки происходят через заранее 
    известные промежутки времени, и считать их событиями нельзя.
    
    Если уменьшить величину порога до $0.008$, 
    то оба метода будут детектировать события.
    Но при этом, некоторые из них останутся без внимания,
    например, спад перед сороковой секундой.
    Однако, при дальнейшем понижении порога в качестве событий 
    будут определять паразитные шумы калибровки.
\end{note-frame}

\begin{image-frame}{Съёмка с БПЛА над побережьем Тулума ($T_{const} = 0.008$)}
    \includegraphics[height=8.2cm]%
    {img/video/example/threshold/static/sad-ffmpeg-tulum-0008.pdf}
\end{image-frame}




