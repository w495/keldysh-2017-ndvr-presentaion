\begin{frame}{Как свести к одномерному случаю}

\begin{small}
\begin{columns}[t]
    \begin{column}{0.5\textwidth}
        \begin{blue-box}{Дана последовательность кадров}
            \[
                F(t) =  F_{rgb,t}
                \Ta
            \]
            \[
            F_{t} = \left(
                \begin{array}{ccc}
                    x_{1,1}, & \cdots & x_{1,m} \\
                    \cdots   & \cdots & \cdots  \\
                    x_{n,1}, & \cdots & x_{n,m} \\
                \end{array} 
            \right)_t
            \]
            \[
            x_{i,j} = \left\lbrace 
                r_{i,j}; g_{i,j}; b_{i,j} 
            \right\rbrace 
            \in \mathbb{R}^{3}_{[0,1]};
            \]
        \end{blue-box}
        \begin{footnotesize}
            \begin{green-box}{\unskip}
                Работаем только с визуальным рядом
                \begin{itemize}
                    \item может быть видео без звука;
                    \item звука без картинки в видео не бывает.
                \end{itemize}
            \end{green-box}
        \end{footnotesize}
    \end{column}
    \begin{column}{0.5\textwidth}
        \begin{orange-box}{Для каждого кадра}
            Значение по \textit{выбранной} норме,
            например:
            \[
                \left\| F_t \right\|_{L_1} 
                    = \dfrac{1}{n \cdot m} 
                        \sum\limits_{i=0}^{n}
                            \sum\limits_{j=0}^{m}
                                S(x_{i,j}), \text{где}\quad
            \]
            \[
                S: 
                    \mathbb{R}^{3}_{[0,1]} 
                        \longmapsto 
                            \mathbb{R}_{[0,1]}
            \]
        \end{orange-box}
        \begin{footnotesize}
            \begin{green-box}{\unskip}
                \textcolor{grassgreen}{
                    Например, \quad 
                    $\color{grassgreen} S(x) = |Y_{Y'P_rP_b}| $
                }
            \end{green-box}
            \begin{orange-box}{\unskip}
                Далее, строим ряд
                \[
                    F_{L_{1}}(t) = \left\| F_{t} \right\|_{L_1} 
                \]
                и пытаемся найти в нем <<аномалии>>.
            \end{orange-box}
        \end{footnotesize}
    \end{column}
\end{columns}
\end{small}

\end{frame}
